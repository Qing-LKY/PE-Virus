\documentclass[UTF8]{ctexart}
\ctexset{section/format=\Large\bfseries}

% Packages
% font
\usepackage{fontspec}
\setmonofont{Fira Code}[
Contextuals=Alternate  % Activate the calt feature
]
% tikz
\usepackage{tikz}
\usepackage{graphicx}
% listing
\usepackage{listings}
\usepackage{lstfiracode}
% margins
\usepackage{geometry}
\geometry{a4paper,  scale=0.7}
% page header and footer
\usepackage{fancyhdr}
\usepackage{float}
\pagestyle{fancy}
\rhead{}

% For table alignment
\usepackage{array}

\let\oldparagraph\paragraph
\renewcommand{\paragraph}[1]{\oldparagraph{#1} \mbox{}\\}

\lstset{
	columns=fixed,
	style=FiraCodeStyle,
	basicstyle=\small\ttfamily,
	backgroundcolor=\color[RGB]{244,244,244}, 		  % 设定背景颜色:浅晦涩,近透明
	keywordstyle=\color[RGB]{40,40,255},					 % 设定关键字颜色
	commentstyle=\color[RGB]{0,96,96},                        % 设置代码注释的格式
	stringstyle=\color[RGB]{128,0,0},                                % 设置字符串格式
}

\begin{document}
	\title{PE病毒实验}
	\author{}
	\date{\vspace{-8ex}}
	\maketitle
	\begin{table}[h]
		\centering
		
		\begin{tabular}{|p{2.5cm}<{\centering}|p{4.5cm}<{\centering}|p{3cm}<{\centering}|p{2.5cm}<{\centering}|}
			\hline
			\textbf{课程名称} & 软件安全         & \textbf{实验日期} & 2022.10.29   \\ \hline
			\textbf{姓名}   & \textbf{学号}   & \textbf{专业}   & \textbf{班级} \\ \hline
			李心杨           & 2020302181022 & 信息安全          & 2           \\
			林锟扬           & 2020302181032 & 信息安全          & 2           \\
			上官景威      &  2020302181069 & 信息安全          & 3           \\ \hline
		\end{tabular}
	\end{table}
	\tableofcontents
	\pagebreak
	
	\section{实验目的及实验内容}
	\subsection{实验目的}
	加深理解PE文件格式,掌握基本PE病毒的编写技术。
	\subsection{实验任务}
	编写一个PE文件传染程序infect.exe,功能要求如下:
	\begin{enumerate}
		\item
		infect.exe运行后,向同目录下的某个Windows可执行程序(下称目标程序,建议找一个免安装的绿色程序,以方便测试。),植入“病毒载荷”代码。
		\item infect.exe不能重复传染目标程序。
		
		\item【初级任务】目标程序被植入“病毒载荷”后,具备如下行为:一旦执行,就会在其同目录下查找是否有.txt文件,如果有,则任选一个,将其内容复制到另一个位置,文件名为本组某个同学的学号。
		\item【进阶任务】在完成初级任务的情况下,增加如下病毒行为:在其同目录下查找是否有.exe文件,如果有,则传染之。
		\item【自定义任务】在进阶任务的基础上,增加后门,支持远程执行powershell指令。
	\end{enumerate}
	\section{项目说明}
	\subsection{功能概述}
	
	\subsection{编译及测试方式}
	本项目在Windows上开发,使用msvc作为编译器,编译项目需要下载Windows
	SDK。实验过程中用到的编译器及库版本如表~\ref{tab:environment} 所示。
	\begin{table}[h]
		\label{tab:environment}
		\centering
		\begin{tabular}{|c|c|}
			\hline
			软件名&版本信息 \\ \hline 
			Windows 10& 20H2 OS Build 19042.1706 \\
			cl&19.33.31630(x86) \\
			\hline
		\end{tabular}
	\end{table}
	\section{原理及代码解析}
	\subsection{劫持程序入口}
	\subsection{获取动态库函数地址}
	\subsection{使用WinAPI实现传染}
	\subsection{自我复制}
	\subsection{远程指令执行}
	\section{总结}
\end{document}
